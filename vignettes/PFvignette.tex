\documentclass[]{article}
\usepackage{lmodern}
\usepackage{amssymb,amsmath}
\usepackage{ifxetex,ifluatex}
\usepackage{fixltx2e} % provides \textsubscript
\ifnum 0\ifxetex 1\fi\ifluatex 1\fi=0 % if pdftex
  \usepackage[T1]{fontenc}
  \usepackage[utf8]{inputenc}
\else % if luatex or xelatex
  \ifxetex
    \usepackage{mathspec}
  \else
    \usepackage{fontspec}
  \fi
  \defaultfontfeatures{Ligatures=TeX,Scale=MatchLowercase}
\fi
% use upquote if available, for straight quotes in verbatim environments
\IfFileExists{upquote.sty}{\usepackage{upquote}}{}
% use microtype if available
\IfFileExists{microtype.sty}{%
\usepackage{microtype}
\UseMicrotypeSet[protrusion]{basicmath} % disable protrusion for tt fonts
}{}
\usepackage[margin=1in]{geometry}
\usepackage{hyperref}
\hypersetup{unicode=true,
            pdftitle={PF Vignette},
            pdfauthor={David Siev, Christopher H. Tong},
            pdfborder={0 0 0},
            breaklinks=true}
\urlstyle{same}  % don't use monospace font for urls
\usepackage{natbib}
\bibliographystyle{asa}
\usepackage{color}
\usepackage{fancyvrb}
\newcommand{\VerbBar}{|}
\newcommand{\VERB}{\Verb[commandchars=\\\{\}]}
\DefineVerbatimEnvironment{Highlighting}{Verbatim}{commandchars=\\\{\}}
% Add ',fontsize=\small' for more characters per line
\usepackage{framed}
\definecolor{shadecolor}{RGB}{248,248,248}
\newenvironment{Shaded}{\begin{snugshade}}{\end{snugshade}}
\newcommand{\KeywordTok}[1]{\textcolor[rgb]{0.13,0.29,0.53}{\textbf{{#1}}}}
\newcommand{\DataTypeTok}[1]{\textcolor[rgb]{0.13,0.29,0.53}{{#1}}}
\newcommand{\DecValTok}[1]{\textcolor[rgb]{0.00,0.00,0.81}{{#1}}}
\newcommand{\BaseNTok}[1]{\textcolor[rgb]{0.00,0.00,0.81}{{#1}}}
\newcommand{\FloatTok}[1]{\textcolor[rgb]{0.00,0.00,0.81}{{#1}}}
\newcommand{\ConstantTok}[1]{\textcolor[rgb]{0.00,0.00,0.00}{{#1}}}
\newcommand{\CharTok}[1]{\textcolor[rgb]{0.31,0.60,0.02}{{#1}}}
\newcommand{\SpecialCharTok}[1]{\textcolor[rgb]{0.00,0.00,0.00}{{#1}}}
\newcommand{\StringTok}[1]{\textcolor[rgb]{0.31,0.60,0.02}{{#1}}}
\newcommand{\VerbatimStringTok}[1]{\textcolor[rgb]{0.31,0.60,0.02}{{#1}}}
\newcommand{\SpecialStringTok}[1]{\textcolor[rgb]{0.31,0.60,0.02}{{#1}}}
\newcommand{\ImportTok}[1]{{#1}}
\newcommand{\CommentTok}[1]{\textcolor[rgb]{0.56,0.35,0.01}{\textit{{#1}}}}
\newcommand{\DocumentationTok}[1]{\textcolor[rgb]{0.56,0.35,0.01}{\textbf{\textit{{#1}}}}}
\newcommand{\AnnotationTok}[1]{\textcolor[rgb]{0.56,0.35,0.01}{\textbf{\textit{{#1}}}}}
\newcommand{\CommentVarTok}[1]{\textcolor[rgb]{0.56,0.35,0.01}{\textbf{\textit{{#1}}}}}
\newcommand{\OtherTok}[1]{\textcolor[rgb]{0.56,0.35,0.01}{{#1}}}
\newcommand{\FunctionTok}[1]{\textcolor[rgb]{0.00,0.00,0.00}{{#1}}}
\newcommand{\VariableTok}[1]{\textcolor[rgb]{0.00,0.00,0.00}{{#1}}}
\newcommand{\ControlFlowTok}[1]{\textcolor[rgb]{0.13,0.29,0.53}{\textbf{{#1}}}}
\newcommand{\OperatorTok}[1]{\textcolor[rgb]{0.81,0.36,0.00}{\textbf{{#1}}}}
\newcommand{\BuiltInTok}[1]{{#1}}
\newcommand{\ExtensionTok}[1]{{#1}}
\newcommand{\PreprocessorTok}[1]{\textcolor[rgb]{0.56,0.35,0.01}{\textit{{#1}}}}
\newcommand{\AttributeTok}[1]{\textcolor[rgb]{0.77,0.63,0.00}{{#1}}}
\newcommand{\RegionMarkerTok}[1]{{#1}}
\newcommand{\InformationTok}[1]{\textcolor[rgb]{0.56,0.35,0.01}{\textbf{\textit{{#1}}}}}
\newcommand{\WarningTok}[1]{\textcolor[rgb]{0.56,0.35,0.01}{\textbf{\textit{{#1}}}}}
\newcommand{\AlertTok}[1]{\textcolor[rgb]{0.94,0.16,0.16}{{#1}}}
\newcommand{\ErrorTok}[1]{\textcolor[rgb]{0.64,0.00,0.00}{\textbf{{#1}}}}
\newcommand{\NormalTok}[1]{{#1}}
\usepackage{graphicx,grffile}
\makeatletter
\def\maxwidth{\ifdim\Gin@nat@width>\linewidth\linewidth\else\Gin@nat@width\fi}
\def\maxheight{\ifdim\Gin@nat@height>\textheight\textheight\else\Gin@nat@height\fi}
\makeatother
% Scale images if necessary, so that they will not overflow the page
% margins by default, and it is still possible to overwrite the defaults
% using explicit options in \includegraphics[width, height, ...]{}
\setkeys{Gin}{width=\maxwidth,height=\maxheight,keepaspectratio}
\IfFileExists{parskip.sty}{%
\usepackage{parskip}
}{% else
\setlength{\parindent}{0pt}
\setlength{\parskip}{6pt plus 2pt minus 1pt}
}
\setlength{\emergencystretch}{3em}  % prevent overfull lines
\providecommand{\tightlist}{%
  \setlength{\itemsep}{0pt}\setlength{\parskip}{0pt}}
\setcounter{secnumdepth}{0}
% Redefines (sub)paragraphs to behave more like sections
\ifx\paragraph\undefined\else
\let\oldparagraph\paragraph
\renewcommand{\paragraph}[1]{\oldparagraph{#1}\mbox{}}
\fi
\ifx\subparagraph\undefined\else
\let\oldsubparagraph\subparagraph
\renewcommand{\subparagraph}[1]{\oldsubparagraph{#1}\mbox{}}
\fi

%%% Use protect on footnotes to avoid problems with footnotes in titles
\let\rmarkdownfootnote\footnote%
\def\footnote{\protect\rmarkdownfootnote}

%%% Change title format to be more compact
\usepackage{titling}

% Create subtitle command for use in maketitle
\newcommand{\subtitle}[1]{
  \posttitle{
    \begin{center}\large#1\end{center}
    }
}

\setlength{\droptitle}{-2em}
  \title{PF Vignette}
  \pretitle{\vspace{\droptitle}\centering\huge}
  \posttitle{\par}
  \author{David Siev, Christopher H. Tong}
  \preauthor{\centering\large\emph}
  \postauthor{\par}
  \predate{\centering\large\emph}
  \postdate{\par}
  \date{January 4, 2012}


\begin{document}
\maketitle

\subsection{Introduction}\label{introduction}

The \texttt{PF} package is a collection of functions related to
estimating prevented fraction, \(PF=1-RR\) , where
\(RR={{{\pi}_{2}}}/{{{\pi}_{1}}}\;\) .

\subsubsection{Technical notes}\label{technical-notes}

\emph{Optimization}. Unless otherwise stated, optimization is by the DUD
algorithm {[}RJ78{]}.

\emph{Level tested}. The help files indicate the level of testing
undergone by each function. In some cases that is a subjective
judgement, since most of these functions were originally tested in SPlus
and have been ported to R more recently

\subsection{Score based methods}\label{score-based-methods}

\subsubsection{The score statistic}\label{the-score-statistic}

Confidence intervals for the risk ratio may be based on the score
statistic {[}Koop84, MN85{]},
\[\frac{{{{\hat{\pi }}}_{2}}-{{\rho }_{0}}{{{\hat{\pi }}}_{1}}}{\sqrt{\left( {{{{\tilde{\pi }}}_{2}}(1-{{{\tilde{\pi }}}_{2}})}/{{{n}_{2}}}\; \right)+\rho _{0}^{2}\left( {{{{\tilde{\pi }}}_{1}}(1-{{{\tilde{\pi }}}_{1}})}/{{{n}_{1}}}\; \right)}}\]
where hat indicates the MLE and tilde indicates the MLE under the
restriction that \(\rho ={{\rho }_{0}}\).

\subsection{Asymptotic intervals}\label{asymptotic-intervals}

\texttt{RRsc()} estimates asymptotic confidence intervals for the risk
ratio or prevented fraction based on the score statistic. Interval
estimates are returned for three estimators. The score method was
originally introduced by {[}Koop84{]}. Gart and Nam's modification
includes a skewness correction {[}GN88{]}. The method of {[}MN85{]} is a
version made slightly more conservative than Koopman's by including a
factor of \({(N-1)}/{N}\;\). \pagebreak

\begin{Shaded}
\begin{Highlighting}[]
\KeywordTok{require}\NormalTok{(PF)}
\KeywordTok{RRsc}\NormalTok{(}\KeywordTok{c}\NormalTok{(}\DecValTok{4}\NormalTok{,}\DecValTok{24}\NormalTok{,}\DecValTok{12}\NormalTok{,}\DecValTok{28}\NormalTok{))}
\end{Highlighting}
\end{Shaded}

\begin{verbatim}
## 
## PF 
## 95% interval estimates
## 
##                 PF     LL    UL
## MN method    0.611 0.0251 0.857
## score method 0.611 0.0328 0.855
## skew corr    0.611 0.0380 0.876
\end{verbatim}

\bigskip
Starting estimates for the algorithm are obtained by the modified Katz
method (log method with 0.5 added to each cell). Both forms of the Katz
estimate may be retrieved from the returned object using
\texttt{RRsc()\textbackslash{}\$estimate}.

\subsection{Exact intervals}\label{exact-intervals}

These methods give intervals that are `exact' in the sense that they are
based on the actual sampling distribution rather than an approximation
to it. The score statistic is used to select the \(2 \times 2\) tables
that would fall in the tail area, and the binomial probability is
estimated over the tail area by taking the maximum over the nuisance
parameter. The search over the tail area is made more efficient by the
Berger-Boos correction {[}BB94{]}.

\texttt{RRtosst()} gives intervals obtained by inverting two one-sided
score tests; \texttt{RRotsst()} gives intervals obtained by inverting
one two-sided score test. \texttt{RRtosst()} is thus more conservative,
preserving at least \({\alpha }/{2}\;\) in each tail. {[}AM01{]} discuss
the properties and relative benefits of the two approaches. The price of
exactnesss is conservatism, due to the discreteness of the binomial
distribution {[}AA01{]}. This means that the actual coverage of the
confidence interval does not exactly conform to the nominal coverage,
but it will not be less than it. (See also {[}AA03{]}.) Both functions
use a simple step search algorithm.

\begin{Shaded}
\begin{Highlighting}[]
\KeywordTok{RRotsst}\NormalTok{(}\KeywordTok{c}\NormalTok{(}\DecValTok{4}\NormalTok{,}\DecValTok{24}\NormalTok{,}\DecValTok{12}\NormalTok{,}\DecValTok{28}\NormalTok{))}
\end{Highlighting}
\end{Shaded}

\begin{verbatim}
## 
## PF 
## 95% interval estimates
## 
##     PF     LL     UL 
## 0.6111 0.0148 0.8519
\end{verbatim}

\begin{Shaded}
\begin{Highlighting}[]
\KeywordTok{RRtosst}\NormalTok{(}\KeywordTok{c}\NormalTok{(}\DecValTok{4}\NormalTok{,}\DecValTok{24}\NormalTok{,}\DecValTok{12}\NormalTok{,}\DecValTok{28}\NormalTok{))}
\end{Highlighting}
\end{Shaded}

\begin{verbatim}
## 
## PF 
## 95% interval estimates
## 
##    PF    LL    UL 
## 0.611 0.012 0.902
\end{verbatim}

\section{Stratified designs}

Methods for estimating a common \emph{RR} from stratified or clustered
designs depend on homogeneity with respect to the common parameter.

\subsection{Gart-Nam score method \label{sec:gart}}

{[}Gart85{]} and {[}GN88{]} derived a score statistic for a common
estimate of \emph{RR} from designs with multiple independent strata, and
they showed that it is identical to one proposed by {[}Rad65{]} from a
different perspective.

\texttt{RRstr()} provides confidence intervals and a homogeneity test
based on Gart's statistic.

Data may be input two ways, either using a formula and data frame, or as
a matrix.

\begin{Shaded}
\begin{Highlighting}[]
\KeywordTok{RRstr}\NormalTok{(}\KeywordTok{cbind}\NormalTok{(y,n) ~}\StringTok{ }\NormalTok{tx +}\StringTok{ }\KeywordTok{cluster}\NormalTok{(clus), Table6 , }\DataTypeTok{pf =} \NormalTok{F)}
\end{Highlighting}
\end{Shaded}

\begin{verbatim}
## [1] "ok"
## [1] "end"
\end{verbatim}

\begin{verbatim}
## 
## Test of homogeneity across clusters
## 
## stat  0.954 
## df    3 
## p     0.812 
## 
## RR 
## 95% interval estimates
## 
##             RR   LL   UL
## starting  2.66 1.37 5.18
## mle       2.65 1.39 5.03
## skew corr 2.65 1.31 5.08
\end{verbatim}

\begin{Shaded}
\begin{Highlighting}[]
\CommentTok{# Data matrix input:}
\CommentTok{# RRstr(Y = table6, pf = F)}
\end{Highlighting}
\end{Shaded}

\subsection{Mantel-Haenszel estimator}

A widely-used heuristic method for sparse frequency tables is the
weighted average approach of {[}MH59{]}.\footnote{[KLK88]} review the
Mantel-Haenzel approach and point out its relationship to a method
proposed by {[}Coc54{]}, which was the basis of Rhadakrishnan's method
{[}Rad65{]}, alluded to in section\textasciitilde{}\ref{sec:gart}.\} MH
interval estimates are based on the asymptotic normality of the log of
the risk ratio. \texttt{RRmh()} utilizes the variance estimator given by
{[}GR85{]} for sparse strata. The resulting asymptotic estimator is
consistent for both the case of sparse strata where the number of strata
is assumed increasing, and the case of limited number of strata where
the stratum size is assumed increasing. In the latter case, however, it
is less efficient than maximum likelihood {[}AH00, GR85{]}. Additional
discussion may be found in Section 4.3.1 {[}Lachin00{]}, {[}LSKK05{]},
and {[}SO06{]}.
\footnote{SAS Proc FREQ provides MH interval estimates of \emph{RR}. The other estimator calculated by Proc FREQ, which it calls
``logit,'' is actually a weighted least squares estimator [Lachin00] that has a demonstrable and severe
bias for sparse data [GR85].  It should be avoided.}

\begin{Shaded}
\begin{Highlighting}[]
\KeywordTok{RRmh}\NormalTok{(}\KeywordTok{cbind}\NormalTok{(y,n) ~}\StringTok{ }\NormalTok{tx +}\StringTok{ }\KeywordTok{cluster}\NormalTok{(clus), Table6 , }\DataTypeTok{pf =} \NormalTok{F)}
\end{Highlighting}
\end{Shaded}

\begin{verbatim}
## [1] "ok"
## [1] "end"
\end{verbatim}

\begin{verbatim}
## 
## RR 
## 95% interval estimates
## 
##   RR   LL   UL 
## 2.67 1.37 5.23
\end{verbatim}

\begin{Shaded}
\begin{Highlighting}[]
\CommentTok{# Data matrix input:}
\CommentTok{# RRmh(Y = table6, pf = F)}
\end{Highlighting}
\end{Shaded}

\subsection{Examples}

For a fuller set of examples, see the vignette
\emph{Examples for Stratified Designs}.

\section{Model based intervals}\subsection{Logistic regression estimates}

Intervals may be estimated from logistic regression models with
\texttt{RRor()}. It takes the fit of a \texttt{glm()} object and
estimates the intervals by the delta method.

\begin{Shaded}
\begin{Highlighting}[]
\NormalTok{bird.fit <-}\StringTok{ }\KeywordTok{glm}\NormalTok{(}\KeywordTok{cbind}\NormalTok{(y,n-y) ~}\StringTok{ }\NormalTok{tx -}\StringTok{ }\DecValTok{1}\NormalTok{, binomial, bird)}
\KeywordTok{RRor}\NormalTok{(bird.fit)}
\end{Highlighting}
\end{Shaded}

\begin{verbatim}
## 
## 95% t intervals on 4 df
## 
## PF 
##      PF      LL      UL 
##  0.5000 -0.0406  0.7598 
## 
##       mu.hat    LL    UL
## txcon  0.733 0.896 0.466
## txvac  0.367 0.624 0.168
\end{verbatim}

\subsection{Estimating the dispersion parameter}

The binomial GLM weights are
\[\frac{\hat{\pi }(1-\hat{\pi })}{a(\hat{\varphi })/n\;}\] where
\(a(\hat{\varphi })\) is a function of the dispersion parameter.

\subsubsection{Dispersion parameter $\varphi$}

A simple estimator of the dispersion parameter, \(\varphi\), may be
estimated by the method of moments {[}Wed74{]}. It is given by
\texttt{phiWt()}. This form of the dispersion parameter has
\(a(\varphi)=\varphi\), and \(\varphi\) is estimated by
\({X^2}/{df}\;\), the Pearson statistic divided by the degrees of
freedom.

Note that \(\varphi\) is the same estimator as may be obtained by the
\texttt{quasibinomial} family in \texttt{glm()} which is, in fact, what
is used by \texttt{phiWt()} to reweight the original fit:

\begin{Shaded}
\begin{Highlighting}[]
\KeywordTok{phiWt}\NormalTok{(bird.fit, }\DataTypeTok{fit.only =} \NormalTok{F)$phi}
\end{Highlighting}
\end{Shaded}

\begin{verbatim}
##      all 
## 2.471592
\end{verbatim}

\begin{Shaded}
\begin{Highlighting}[]
\KeywordTok{summary}\NormalTok{(}\KeywordTok{update}\NormalTok{(bird.fit, }\DataTypeTok{family =} \NormalTok{quasibinomial))$disp}
\end{Highlighting}
\end{Shaded}

\begin{verbatim}
## [1] 2.471592
\end{verbatim}

\bigskip
\texttt{phiWt()} makes it easy to estimate \emph{PF} intervals with a
single command. \bigskip

\begin{Shaded}
\begin{Highlighting}[]
\CommentTok{# model weighted by phi hat}
\KeywordTok{RRor}\NormalTok{(}\KeywordTok{phiWt}\NormalTok{(bird.fit))}
\end{Highlighting}
\end{Shaded}

\begin{verbatim}
## 
## 95% t intervals on 4 df
## 
## PF 
##     PF     LL     UL 
##  0.500 -0.583  0.842 
## 
##       mu.hat    LL     UL
## txcon  0.733 0.943 0.3121
## txvac  0.367 0.752 0.0997
\end{verbatim}

\bigskip
It also allows different estimates of \({\hat{\varphi}}\) for specified
subsets of the data. \bigskip

\begin{Shaded}
\begin{Highlighting}[]
\CommentTok{# model with separate phi for vaccinates and controls}
\KeywordTok{RRor}\NormalTok{(}\KeywordTok{phiWt}\NormalTok{(bird.fit, }\DataTypeTok{subset.factor =} \NormalTok{bird$tx))}
\end{Highlighting}
\end{Shaded}

\begin{verbatim}
## 
## 95% t intervals on 4 df
## 
## PF 
##     PF     LL     UL 
##  0.500 -0.645  0.848 
## 
##       mu.hat    LL     UL
## txcon  0.733 0.938 0.3330
## txvac  0.367 0.767 0.0925
\end{verbatim}

\bigskip
If you want to subtract a degree of freedom for each additional
parameter, you can do that by entering the degrees of freedom as an
argument to \texttt{RRor()}. \bigskip

\begin{Shaded}
\begin{Highlighting}[]
\CommentTok{# subtract 2 degrees of freedom}
\KeywordTok{RRor}\NormalTok{(}\KeywordTok{phiWt}\NormalTok{(bird.fit, }\DataTypeTok{subset.factor =} \NormalTok{bird$tx), }\DataTypeTok{degf =} \DecValTok{2}\NormalTok{)}
\end{Highlighting}
\end{Shaded}

\begin{verbatim}
## 
## 95% t intervals on 2 df
## 
## PF 
##     PF     LL     UL 
##  0.500 -2.164  0.921 
## 
##       mu.hat    LL     UL
## txcon  0.733 0.975 0.1635
## txvac  0.367 0.895 0.0377
\end{verbatim}

\subsubsection{Dispersion parameter $\tau$}

When overdispersion is due to intra-cluster correlation, it may make
sense to estimate the dispersion as a function of the intra-cluster
correlation parameter \(\tau\). In other words,
\({a({\varphi }_{ij})}=1+{{\tau }_{j}}({{n}_{ij}}-1)\). \texttt{tauWt()}
does this using the Williams procedure {[}Wil82\}. \bigskip

\begin{Shaded}
\begin{Highlighting}[]
\CommentTok{# model weighted using tau hat}
\KeywordTok{RRor}\NormalTok{(}\KeywordTok{tauWt}\NormalTok{(bird.fit, }\DataTypeTok{subset.factor =} \NormalTok{bird$tx))}
\end{Highlighting}
\end{Shaded}

\begin{verbatim}
## 
## 95% t intervals on 4 df
## 
## PF 
##     PF     LL     UL 
##  0.500 -0.645  0.848 
## 
##       mu.hat    LL     UL
## txcon  0.733 0.938 0.3330
## txvac  0.367 0.767 0.0925
\end{verbatim}

\bigskip
In this example the \texttt{tauWt()} estimates are the same as the
\texttt{phiWt()} estimates. That is because the cluster sizes are all
the same. Let's see what happens if we modify the \texttt{bird} data
set. The \texttt{birdm} data set has the same cluster fractions but
differing cluster sizes. \bigskip

\begin{Shaded}
\begin{Highlighting}[]
\CommentTok{# different cluster sizes, same cluster fractions}
\NormalTok{birdm.fit <-}\StringTok{ }\KeywordTok{glm}\NormalTok{(}\KeywordTok{cbind}\NormalTok{(y,n-y) ~}\StringTok{ }\NormalTok{tx -}\StringTok{ }\DecValTok{1}\NormalTok{, binomial, birdm)}
\KeywordTok{RRor}\NormalTok{(}\KeywordTok{tauWt}\NormalTok{(birdm.fit, }\DataTypeTok{subset.factor =} \NormalTok{birdm$tx))}
\end{Highlighting}
\end{Shaded}

\begin{verbatim}
## 
## 95% t intervals on 4 df
## 
## PF 
##     PF     LL     UL 
##  0.490 -0.605  0.838 
## 
##       mu.hat    LL    UL
## txcon  0.737 0.942 0.328
## txvac  0.376 0.764 0.101
\end{verbatim}

\bigskip
Note that increasing cluster size can make things worse when there is
intra-cluster correlation.

\bigskip
Now let's compare the weights from \texttt{phiWt()} and \texttt{tauWt()}
with unequal cluster sizes. In the output below, \emph{w} represents
\(1/a(\hat{\varphi })\;\) and \emph{nw} is \(n/a(\hat{\varphi })\;\)
\bigskip

\begin{Shaded}
\begin{Highlighting}[]
\CommentTok{# Compare phi and tau weights}
\CommentTok{#  }
\NormalTok{phi.wts <-}\KeywordTok{phiWt}\NormalTok{(birdm.fit,}\DataTypeTok{fit.only =} \NormalTok{F, }\DataTypeTok{subset.factor =} \NormalTok{birdm$tx)$weights}
\NormalTok{tau.wts <-}\StringTok{ }\KeywordTok{tauWt}\NormalTok{(birdm.fit,}\DataTypeTok{fit.only =} \NormalTok{F, }\DataTypeTok{subset.factor =} \NormalTok{birdm$tx)$weights}
\NormalTok{w <-}\StringTok{ }\KeywordTok{cbind}\NormalTok{(}\DataTypeTok{w.phi=}\NormalTok{phi.wts,}\DataTypeTok{w.tau=}\NormalTok{tau.wts,}\DataTypeTok{nw.phi=}\NormalTok{phi.wts*birdm$n,}
        \DataTypeTok{nw.tau=}\NormalTok{tau.wts*birdm$n)}
\KeywordTok{print}\NormalTok{(}\KeywordTok{cbind}\NormalTok{(birdm[,}\KeywordTok{c}\NormalTok{(}\DecValTok{3}\NormalTok{,}\DecValTok{1}\NormalTok{,}\DecValTok{2}\NormalTok{)],}\KeywordTok{round}\NormalTok{(w, }\DecValTok{2}\NormalTok{)), }\DataTypeTok{row.names=}\NormalTok{F)}
\end{Highlighting}
\end{Shaded}

\begin{verbatim}
##   tx  y  n w.phi w.tau nw.phi nw.tau
##  vac  1 10  0.32  0.35   3.20   3.55
##  vac  8 20  0.32  0.21   6.40   4.13
##  vac  9 15  0.32  0.26   4.80   3.92
##  con  8 16  0.21  0.27   3.39   4.33
##  con  8 10  0.21  0.38   2.12   3.82
##  con 27 30  0.21  0.16   6.36   4.84
\end{verbatim}

\bigskip
Look at the last two rows. Note that the \texttt{nw.phi} are directly
proportional to \emph{n} within treatment group, while the
\texttt{nw.tau} are not. With intra-cluster correlation, increasing
cluster size does not give a corresponding increase in information.
\bigskip

\subsection{Rao-Scott weights}

{[}RS92\} give a method of weighting clustered binomial observations
based on the variance inflation due to clustering. They relate their
approach to the concepts of design effect and effective sample size
familiar in survey sampling, and they illustrate its use in a variety of
contexts. \texttt{rsbWt()} implements it in the same manner as
\texttt{phiWt()} and \texttt{tauWt()}. For more general use, the
function \texttt{rsb()} just returns the design effect estimates and the
weights.

\begin{Shaded}
\begin{Highlighting}[]
\CommentTok{# model weighted with Rao-Scott weights}
\KeywordTok{RRor}\NormalTok{(}\KeywordTok{rsbWt}\NormalTok{(birdm.fit, }\DataTypeTok{subset.factor =} \NormalTok{birdm$tx))}
\end{Highlighting}
\end{Shaded}

\begin{verbatim}
## 
## 95% t intervals on 4 df
## 
## PF 
##     PF     LL     UL 
##  0.479 -0.314  0.793 
## 
##       mu.hat    LL    UL
## txcon  0.768 0.960 0.311
## txvac  0.400 0.717 0.149
\end{verbatim}

\begin{Shaded}
\begin{Highlighting}[]
\CommentTok{# just the design effect estimates}
\KeywordTok{rsb}\NormalTok{(birdm$y, birdm$n, birdm$tx)$d}
\end{Highlighting}
\end{Shaded}

\begin{verbatim}
##      con      vac 
## 5.137107 2.500000
\end{verbatim}

\section{Likelihood based intervals}

The \texttt{RRlsi()} function estimates likelihood support intervals for
\emph{RR} by the profile likelihood Section 7.6{[}Roy97{]}.

Likelihood support intervals are usually formed based on the desired
likelihood ratio, \({1}/{k}\;\), often \({1}/{8}\;\) or \({1}/{32}\;\).
Under some conditions the log likelihood ratio may follow the chi-square
distribution. If so, then
\(\alpha =1-{{F}_{{{\chi }^{2}}}}\left( 2\log (k),1 \right)\).
\texttt{RRsc()} will make the conversion from \(\alpha\) to \emph{k}
with the argument \texttt{use.alpha = T}.

\bigskip

\begin{Shaded}
\begin{Highlighting}[]
\KeywordTok{RRlsi}\NormalTok{(}\KeywordTok{c}\NormalTok{(}\DecValTok{4}\NormalTok{,}\DecValTok{24}\NormalTok{,}\DecValTok{12}\NormalTok{,}\DecValTok{28}\NormalTok{))}
\end{Highlighting}
\end{Shaded}

\begin{verbatim}
## 
## 1/8 likelihood support interval for PF 
## 
## corresponds to 95.858% confidence
##   (under certain assumptions)
## 
## PF 
##     PF     LL     UL 
## 0.6111 0.0168 0.8859
\end{verbatim}

\begin{Shaded}
\begin{Highlighting}[]
\KeywordTok{RRlsi}\NormalTok{(}\KeywordTok{c}\NormalTok{(}\DecValTok{4}\NormalTok{,}\DecValTok{24}\NormalTok{,}\DecValTok{12}\NormalTok{,}\DecValTok{28}\NormalTok{), }\DataTypeTok{use.alpha =} \NormalTok{T)}
\end{Highlighting}
\end{Shaded}

\begin{verbatim}
## 
## 1/6.826 likelihood support interval for PF 
## 
## corresponds to 95% confidence
##   (under certain assumptions)
## 
## PF 
##     PF     LL     UL 
## 0.6111 0.0495 0.8792
\end{verbatim}

\section{Incidence ratio}

The incidence is the number of cases per subject-time. Its distribution
is assumed Poisson. Under certain designs, the incidence ratio
(\emph{IR}) is used as a measure of treatment effect. Correspondingly,
\({{PF}_{IR}}=1-IR\) would be used as a measure of effect for an
intervention that is preventive, such as vaccination. \emph{IR} is also
called incidence density ratio (\emph{IDR}), and that is the term used
in the following functions.

\subsection{Score method}

\texttt{IDRsc()} estimates a confidence interval for the incidence
density ratio using Siev's formula Appendix 1 {[}Siev94{]} based on the
Poisson score
statistic.\footnote{This formula was published in a Japanese journal [Sato88] several years before Siev. See also [GMM03] and [Siev04].}
\[IDR=\widehat{IDR}\left\{ 1+\left( \frac{1}{{{y}_{1}}}+\frac{1}{{{y}_{2}}} \right)\frac{z_{\alpha /2}^{2}}{2}\ \ \pm \ \ \frac{z_{\alpha /2}^{2}}{2{{y}_{1}}{{y}_{2}}}\sqrt{{{y}_{\bullet }}\left( {{y}_{\bullet }}z_{\alpha /2}^{2}+4{{y}_{1}}{{y}_{2}} \right)} \right\}\]

\bigskip

\begin{Shaded}
\begin{Highlighting}[]
\KeywordTok{IDRsc}\NormalTok{(}\KeywordTok{c}\NormalTok{(}\DecValTok{26}\NormalTok{,}\DecValTok{204}\NormalTok{,}\DecValTok{10}\NormalTok{,}\DecValTok{205}\NormalTok{), }\DataTypeTok{pf =} \NormalTok{F)}
\end{Highlighting}
\end{Shaded}

\begin{verbatim}
## 
## IDR 
## 95% interval estimates
## 
##  IDR   LL   UL 
## 2.61 1.28 5.34
\end{verbatim}

\subsection{Likelihood method}

A likelihood support interval for \emph{IDR} may be estimated based on
orthogonal factoring of the reparameterized likelihood. Section 7.2
{[}Roy97{]} \texttt{IDRlsi()} implements this method.

Likelihood support intervals are usually formed based on the desired
likelihood ratio, \({1}/{k}\;\), often \({1}/{8}\;\) or \({1}/{32}\;\).
Under some conditions the log likelihood ratio may follow the chi square
distribution. If so, then
\(\alpha =1-{{F}_{{{\chi }^{2}}}}\left( 2\log (k),1 \right)\).
\texttt{IDRlsi()} will make the conversion from \(\alpha\) tp \emph{k}
with the argument \texttt{use.alpha = T}.

\bigskip

\begin{Shaded}
\begin{Highlighting}[]
\KeywordTok{IDRlsi}\NormalTok{(}\KeywordTok{c}\NormalTok{(}\DecValTok{26}\NormalTok{,}\DecValTok{204}\NormalTok{,}\DecValTok{10}\NormalTok{,}\DecValTok{205}\NormalTok{), }\DataTypeTok{pf =} \NormalTok{F)}
\end{Highlighting}
\end{Shaded}

\begin{verbatim}
## 
## 1/8 likelihood support interval for IDR 
## 
## corresponds to 95.858% confidence
##   (under certain assumptions)
## 
## IDR 
##  IDR   LL   UL 
## 2.61 1.26 5.88
\end{verbatim}

\begin{Shaded}
\begin{Highlighting}[]
\KeywordTok{IDRlsi}\NormalTok{(}\KeywordTok{c}\NormalTok{(}\DecValTok{26}\NormalTok{,}\DecValTok{204}\NormalTok{,}\DecValTok{10}\NormalTok{,}\DecValTok{205}\NormalTok{), }\DataTypeTok{pf =} \NormalTok{F, }\DataTypeTok{use.alpha =} \NormalTok{T)}
\end{Highlighting}
\end{Shaded}

\begin{verbatim}
## 
## 1/6.826 likelihood support interval for IDR 
## 
## corresponds to 95% confidence
##   (under certain assumptions)
## 
## IDR 
##  IDR   LL   UL 
## 2.61 1.30 5.69
\end{verbatim}

\bibliographystyle{asa}\bibliography{PF}

\bibliography{PF}


\end{document}
